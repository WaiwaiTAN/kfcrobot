% 设置 biblatex 额外选项
% \PassOptionsToPackage{gbpub=false, gbtype=false}{biblatex}

% 载入 SJTUThesis 模版
% \documentclass[degree=doctor, zihao=-4, language=english, review]{sjtuthesis}
% \documentclass[degree=master, zihao=-4]{sjtuthesis}
% \documentclass[degree=bachelor, openany, oneside]{sjtuthesis}
\documentclass[degree=course, language=english, openright, twoside]{sjtuthesis}
% 选项
%   degree=[doctor|master|bachelor|course],     % 必选,学位类型
%   language=[chinese|english],                 % 可选(默认:chinese),论文的主要语言
%   bibstyle=[gb7714-2015|gb7714-2015ay|ieee],  % 可选(默认:gb7714-2015),参考文献样式
%   review,                                     % 可选(默认:关闭),盲审模式

% 所有其它可能用到的包都统一放到这里了,可以根据自己的实际添加或者删除。
\usepackage{sjtuthesis}
\definecolor{commentcolor}{RGB}{85,139,78}
\definecolor{stringcolor}{RGB}{206,145,108}
\definecolor{keywordcolor}{RGB}{34,34,250}
\definecolor{backcolor}{RGB}{220,220,220}

\usepackage{accsupp}    
\newcommand{\emptyaccsupp}[1]{\BeginAccSupp{ActualText={}}#1\EndAccSupp{}}

\usepackage{listings}
\lstset{                        %高亮代码设置
    language=python,                    %Python语法高亮
    linewidth=0.9\linewidth,            %列表list宽度
    %basicstyle=\ttfamily,              %tt无法显示空格
    commentstyle=\color{commentcolor},  %注释颜色
    keywordstyle=\color{keywordcolor},  %关键词颜色
    stringstyle=\color{stringcolor},    %字符串颜色
    %showspaces=true,                   %显示空格
    numbers=left,                       %行数显示在左侧
    numberstyle=\tiny\emptyaccsupp,     %行数数字格式
    numbersep=5pt,                      %数字间隔
    frame=single,                       %加框
    framerule=0pt,                      %不划线
    escapeinside=@@,                    %逃逸标志
    emptylines=1,                       %
    xleftmargin=3em,                    %list左边距
    backgroundcolor=\color{backcolor},  %列表背景色
    tabsize=4,                          %制表符长度为4个字符
    gobble=4                            %忽略每行代码前4个字符
    }

% 定义图片文件目录与扩展名
\graphicspath{{figure/}}
\DeclareGraphicsExtensions{.pdf,.eps,.png,.jpg,.jpeg}

% 导入参考文献数据库
\addbibresource{bib/thesis.bib}

% 信息录入,必须在导言区进行!
% !TEX root = ../thesis.tex

%TC:ignore

\title{面向肯德基的餐食分装传送装置\\设计说明书}
\author{
    谭正    \quad{}
    耿梓航  \quad{}
    赵启    \quad{}
    汤逸磊  \quad{}
    徐哲    \quad{}
}
\supervisor{盛鑫军}
% \assisupervisor{某某教授}


\degree{工学硕士}
\major{某某专业}
\department{机械与动力工程学院}
\coursename{设计与制造\uppercase\expandafter{\romannumeral2}}
\date{2014年12月17日}
% \fund{国家 973 项目 (No. 2025CB000000) \\ 国家自然科学基金 (No. 81120250000)}
\keywords{肯德基,餐食传送,五自由度机械臂,自适应}

\entitle{A Design Specification for the Food-collecting and Transferring Manipulator for KFC}
\enauthor{Mo Mo}
\ensupervisor{Prof. Mou Mou}
% \enassisupervisor{Prof. Uom Uom}
\endegree{Master of Engineering}
\enmajor{A Very Important Major}
\endepartment{Depart of XXX}
\endate{Dec. 17th, 2014}
% \enfund{National Basic Research Program of China (Grant No. 2025CB000000) \\
%   National Natural Science Foundation of China (Grant No. 81120250000)}
\enkeywords{KFC, food-collecting and transferring, 5 DOF manipulator,adaptive design}

%TC:endignore


% 自定义项目标签名称
% \sjtuSetLabel{
%   listfigure = {图\quad 录},
%   listtable  = {表\quad 录}
% }

\begin{document}

% 无编号内容:中英文论文封面、授权页
\maketitle
% \makeDeclareOriginality[pdf/originality.pdf]
% \makeDeclareAuthorization

% 使用罗马数字对前言编号
\frontmatter

% 摘要
% !TEX root = ../fkcrobot.tex

\begin{abstract}

  目前肯德基门店存在高峰期店员收银与取餐任务重叠,操作效率不高,取餐等候时间长的问题。本项目旨在设计出面向肯德基的餐食分装传送装置,即一个四自由度机械臂加一自适应机械手,为肯德基目前存在的取餐效率不高和一人多职的问题提供一种模块化、成本低、易于在现有基础上改装的解决方案。制作出的装置能够在收到信指令后,转动到指定位置夹取汉堡、饮料、薯条,并放置到指定区域。本项目中设计一自适应结构来防止破坏食物完整;并在结构设计上力求降低重量和尺寸。

  本设计说明书的内容包括引言、功能分解和总体设计思路、机械结构详细设计、模型的加工与装配、电路控制系统、总结与展望、致谢,共八章。本项目通过采访、实地调查确定意义性与可行性;通过理论计算、建模、仿真确定设计方案;通过机械加工与装配制作模型。
 
\end{abstract}

% \begin{enabstract}
%   Currently during rush hours, KFC staff often has to play the role of cashier and food collector at the same time, leading to poor efficiency and long waiting time.This project aims at designing a food-collecting and transferring equipment for KFC,which is a 4 DOF(Degree of Freedom) manipulator with an adaptive hand,providing a modular, low-cost and easily-modified  solution to KFC's current problems. After receiving signals,the equipment can rotate and move its arms to reach hamburgers,soft drinks or french fries,clamp them,and move them to certain locations. This project involves an adaptive design to protect foods' appearance,and seeks to reduce weight and size.
  
%   This design specification contains 8 chapters: introduction,function decomposition and overall design ideas,detailed structure design, physical model making and assembly,circuit control system,summary and prospect, and acknowledgement.This project's significance and feasibility are confirmed by interviews and field visits; final design scheme is based on theoretical calculation, 3D modeling,and simulation;physical model is made through machining and assembly.
% \end{enabstract}

% 目录、插图目录、表格目录
\tableofcontents
\listoffigures
\listoftables
\listofalgorithms

% 主要符号、缩略词对照表
\input{tex/nomenclature}

% 使用阿拉伯数字对正文编号
\mainmatter

% 正文内容
% !TEX root = ../thesis.tex

\chapter{引言}
%引言部分分为立项背景和任务设计书两部分
%引用时使用intro.tex
%作者:汤逸磊
本项目设计“面向肯德基的餐食分装传送装置”。

本章节聚焦立项背景和设计任务书两部分。

\section{立项背景}
本项目的立项是基于对肯德基目前状况的调查后展开。

本项目选择人流量多的东川路地铁站肯德基门店作为调研对象。经过采访店员后,得到目前门店收银与取餐存在的问题有:高峰期人流多忙不过来;人手不少但效率不高;收银与取餐重叠,存在顾此失彼状况。另外,随机抽选五家上海不同处的肯德基,分析其店面布局,总结出以下特点:收银区域与取餐区域分离;食物集中摆放于食物架上,集中而固定;取餐区域后侧即为食物架,中间有一定可操作区域。

可以发现,肯德基目前需要一个能够减轻店员负担,解决店员一人负责多项任务的状况,且适合肯德基店面布局的产品。本项目旨在设计、制造出一个面向肯德基的餐食分装传送装置,尝试为目前存在的问题提供一种解决新丝思路。
\section{项目任务书}


\subsection{题目:面向肯德基的餐食分装传送装置}

\subsection{项目方案简介}
本项目设计一个五自由度机械臂来夹取汉堡、饮料、小食,其包含有:一个云台、一个大臂(1号臂)、一个前臂(2号臂)、一个腕部(3号臂)、一个自适应末端执行机构(手部)。机械臂功能要实现根据收到的指令,机械臂转动、自适应夹取食物,放置食物。

\begin{figure}[h]
  \centering
  \includegraphics[width=8cm]{intro1.png}
  
\end{figure}

\subsection{项目预期目标}
制作出1:1的机械臂实物模型,并能够实现:

(1)云台360°转动

(2)1号臂俯仰90°

(3)2号臂俯仰50°

(4)腕部俯仰90°

(5)手部张合120°

(6)根据输入指令自动转动机械臂夹取食物,并放置于指定区域。
  


\subsection{任务}

(1)云台、1号臂、2号臂、腕部、手部分别的详细设计图纸

(2)重要零件应力分析

(3)各加工部分的零件图

(4)机械臂的总体装配图

(5)机械臂运动学仿真

(6)电机、驱动器型号选择

(7)机械臂实物制作

(8)设计说明书撰写

\subsection{成果呈现方式}
本项目成果最主要呈现方式为1:1实物模型展示,并能现场演示各项功能。其他成果包括本设计说明书。

\subsection{项目起止时间:2019.9.27至2020.1.11}





\input{tex/floats}
\input{tex/math_and_citations}
% !TEX root = ../thesis.tex

\begin{summary}
    经过我们小组成员的不懈努力,我们的机械臂最终还是能够成功夹取东西并移动到指定位置。
    就课程设计的完成度而言,我们自认为还是比较高的,在11日的项目展上也不出意外的斩获了“最佳人气奖”。
    
    但是从细节上来看,我们的装置还有几点问题:
    \begin{enumerate}
        \item   云台直接压在推力球轴承上,因为推力球轴承不能承受径向力的缘故,我们这样设计的云台稍微一点倾覆力都不能承受。如果不把握好重心,稍有不慎就有可能从基座上翻到下来。
        \item   蜗轮和云台主轴之间的配合不够紧密,键的两个工作面和蜗轮键槽的两个工作面间有较大侧隙。这导致传动时有冲击,而且有些位置不能稳定住。
        \item   大臂与前臂相连的动力传递形式设计不合理:在悬伸端布置了一个同步带轮。因为同步带轮在工作时会对轴产生一个较大的径向力,从而把轴弄变形。这个地方应当考虑齿轮等依赖周向力传递动力的高副。
        \item   重心把握不稳妥,手部应该还能减轻很多重量,用亚克力板材做的手臂两板太厚、太重。
    \end{enumerate}

    当然技术上这些问题都是可以被修复的,并且被修复之后我们的成本会进一步降低,所以这些缺陷也正是我们的发展空间所在。

\end{summary}


% 使用英文字母对附录编号
\appendix

% 附录内容,本科学位论文可以用翻译的文献替代。
\input{tex/app_maxwell_equations}
\input{tex/app_flow_chart}

% 文后无编号部分
\backmatter

% 参考资料
\printbibliography[heading=bibintoc]

% 用于盲审的论文需隐去致谢、发表论文、参与项目、申请专利、简历

% 致谢
% !TEX root = ../thesis.tex

%TC:ignore

\begin{acknowledgements}
  汤逸磊:感谢学校提供场地与加工材料,感谢盛鑫军老师和欧阳崛助教的指导与帮助,感谢所有其他帮助过本项目的人们,最后,感谢我的队友:谭正、徐哲、耿梓航、赵启,没有他们的付出,项目无法成功。

  徐哲:本次项目的顺利完成要感谢许多老师和同学对我们的帮助和支持,
  这其中包括助教与课程老师对我们设计上的答疑解惑与宝贵建议,
  工训的值班老师和助管学长在部分加工换届给出的解决方案和设备支持,
  许多其它组的成员为我们出谋划策、借用工具,以及一些论坛网友所提供的电控教学和指南。
  感谢所有给予我们帮助的人们,让我们能够在圆满完成这次项目的同时,积累宝贵的经验。

  赵启:感谢盛鑫军老师的高质量教学和整个项目从始至终宝贵意见和建议的提出,感谢欧阳崛学长的督促,指导与鼓励,感谢机动学院开设这门对学生理论专业知识和实际动手实践能力都有很大提升的课程,感谢学生创新中心陶波老师对于我们加工的帮助以及中心资源的提供,感谢组员们的互相帮助和学习。

  谭正:感谢小组成员们的积极配合,我们项目的成功离不开你们每一个人;感谢盛鑫军老师给我们小组的批评和教诲,我们从中也学到了很多;感谢助教欧阳崛学长对我们项目的点拨和启发,有些细节真的是多亏了学长我们才真正弄清楚。

\end{acknowledgements}

%TC:endignore


% 发表论文、参与项目、申请专利、简历
% 盲审论文中,发表学术论文及参与科研情况等仅以第几作者注明即可,不要出现作者或他人姓名
\input{tex/publications}
\input{tex/projects}
\input{tex/patents}
\input{tex/resume}

% 中文学士学位论文要求在最后有一个英文大摘要,单独编页码,英文学士学位论文不需要
\input{tex/end_english_abstract}

\end{document}
