% 软件  软件分析
%作者:赵启
%上次更新:2020年1月10日 17:20


\chapter{软件分析}

\section{综述}

  {\songti 在整个项目的设计与之后的装配分析中,我们均学习探索并使用了一些软件进行辅助性分析,
 这对于我们进行相关的材料选择、整体设计、寿命分析以及装配的整体合理性都有了很大的帮助。}

 

\section{基于Ansys分析的蜗杆装配}

  {\songti 在实体装配的过程中,我们通过定性分析可知,基座上的蜗杆
  处于较下方的位置且其上装配有最多的轴承支座并和涡轮相啮合,
  因此会承受很大的应力并产生切变。出于对装配体的
  寿命和安全性的考虑,为了探究其切应力和正应力是否超过其屈服强度和切应力的许用值,我们
  利用Ansys进行有限元辅助分析。}
  
  {\songti 在Ansys中,我们根据装配体的结构情况,进行前处理、输入蜗杆的几何模型、对几何
  模型划分网格、根据装配体的实际情况施加载荷(图~\ref{fig:蜗杆的应力与扭矩}),求解后进行输出应力应变云图。}

  {\songti 在一开始我们设计的装配体中,蜗杆中央受到了极大的正应力,中间一部分超过了45号钢的屈服强度,因此极大地影响
  了蜗杆的工作和使用寿命(图~\ref{fig:改进前正应力})。之后,我们对上方装配体进行了一系列改进优化,如机械手部分尽可能
  采用3D打印结构并对其结构进行优化,并增加了支座分担蜗杆的应力,最终获得了较为理想的输出
  结果(图~\ref{fig:正应力})(图~\ref{fig:切应力})(图~\ref{fig:拉伸位移})} 。
   
  {\songti 由云图可以看出,改进后正应力极大降低,切应力也满足45号钢的许用值(除了极其微小的一部分稍微超过之外),能够满足蜗杆的安全工作条件。}
 
   

  \begin{figure}[!htp]
    \centering
    \includegraphics[width=12cm]{stress_and_torque_of_worm.png}
    \bicaption[软件蜗杆的应力与扭矩]{蜗杆的应力与扭矩}{Stress and torque of worm}
    \label{fig:蜗杆的应力与扭矩}
  \end{figure}

  
  \begin{figure}[!htp]
    \centering
    \includegraphics[width=12cm]{normal_stress_before_improvement.png}
    \bicaption[改进前正应力]{改进前正应力}{Normal stress before improvement}
    \label{fig:改进前正应力}
  \end{figure}

  
  \begin{figure}[!htp]
    \centering
    \includegraphics[width=12cm]{normal_stress.jpg}
    \bicaption[正应力]{正应力}{Normal stress}
    \label{fig:正应力}
  \end{figure}

  \begin{figure}[!htp]
    \centering
    \includegraphics[width=12cm]{shear_stress.png}
    \bicaption[切应力]{切应力}{Shear stress}
    \label{fig:切应力}
  \end{figure}

  \begin{figure}[!htp]
    \centering
    \includegraphics[width=12cm]{tensile_displacement.png}
    \bicaption[拉伸位移]{拉伸位移}{Tensile displacement}
    \label{fig:拉伸位移}
  \end{figure}

  \newpage

\section{基于ROS环境的机械臂分析}

{\songti 为了更好地去分析整个装置,我们小组尝试了Robot Operating System机器人操控系统(以下简称ROS),首先,我们下载安装了了Ubuntu系统,并在Ubuntu系统上搭建ROS环境和相关的功能包,之后在SolidWorks软件中搭建六自由度机械臂的坐标系体系,使用sw urdf exporter
 插件生成了通用机器人描述文件(URDF),成功地把我们设计的模型导入进入了ROS系统。

{\songti 通过ROS环境中模拟出的模型,我们获得了机械臂运动的包络面和模拟路径。并通过后续的学习可以了解到在ROS环境中可以实现电控,但由于时间和精力有限所以小组准备在寒假进行进一步探索。}

\begin{figure}[!htp]
  \centering
  \includegraphics[width=12cm]{ros_environment_setup.jpg}
  \bicaption[ROS环境搭建]{ROS环境搭建}{ROS environment construction}
  \label{fig:ROS环境搭建}
\end{figure}